\documentclass{article}

\usepackage[width=170mm,top=25mm,bottom=25mm,bindingoffset=6mm,left=15mm,right=15mm]{geometry}

\usepackage[brazil]{babel}
\usepackage{hyperref}
\usepackage{hyperref}
\hypersetup{
  colorlinks   = true, %Colours links instead of ugly boxes
  urlcolor     = blue, %Colour for external hyperlinks
  linkcolor    = blue,  %Colour of internal links
  citecolor   = red    %Colour of citations
}
\usepackage{url}

\title{Anotações das reuniões}
\author{QuantUSP}

\begin{document}

\maketitle

\section*{Reunião 0 (10/05/2024)}

Os seguintes tópicos foram discutidos nesta reunião:
\begin{itemize}
    \item Apresentação dos objetivos do grupo e dos participantes.
    \item Apresentação das atividades propostas para as próximas reuniões.
        Tentaremos seguir a seguinte ordem de apresentação de tópicos sobre computação quântica:
        \begin{enumerate}
            \item Definições e propriedades básicas do modelo de circuitos quânticos;
            \item Algoritmo de Deutsch-Jozsa;
            \item Algoritmo de Bernstein-Vazirani;
            \item Algoritmo de Simon;
            \item Algoritmo de Grover;
            \item Algoritmo de Shor.
        \end{enumerate}
        Usaremos como bibliografia básica as notas de aulas de Ronald de Wolf~\cite{deWolfLectureNotes},
        disponível em \url{https://arxiv.org/abs/1907.09415}.
        A respeito de bibliografias adicionais,
        recomendamos o livro de Nielsen e Chuang~\cite{NielsenChuangQCQI} para tópicos gerais em computação e informação quântica,
        e o livro de Lipton e Regan~\cite{LiptonReganIQALA} para computação quântica mais focada em algoritmos quânticos.
        Todas essas bibliografias mencionadas têm alguma revisão dos pré-requisitos de álgebra linear.
        Para uma revisão mais completa dos pré-requisitos, recomendamos o livro \emph{The Mathematics of Quantum Mechanics} de Martin Laforest,
        disponível em \url{http://www.stat.ucla.edu/~ywu/linear.pdf}.
\end{itemize}

\section*{Reunião 1 (17/05/2024)}

Os seguintes tópicos do capítulo~1 e Apêndice~A de~\cite{deWolfLectureNotes} foram discutidos nesta reunião:
\begin{itemize}
    \item Definição de qubit e a Regra de Born;
    \item Breve revisão de propriedades importantes do produto interno e do produto tensorial;
    \item Definição de registrador com $n$-qubits;
    \item Apresentação de alguns exemplos de portas quânticas básicas utilizando simulações por meio do Qiskit.
\end{itemize}

\section*{Reunião 2 (24/05/2024)}

Discutiremos diversos exemplos de gates quânticos e medições de sistemas quânticos (capítulos 1 e 2 de~\cite{deWolfLectureNotes}).

\bibliographystyle{plainurl}
\bibliography{./refs.bib}
\end{document}

